\chapter{Respuestas}
\section{Respuestas Guía 1}
\subsection{Ejercicio 1}
\begin{verbatim}
	
Escribir 'Ingrese un numero entero entre 1 y 499(incluyendo 1 y 499)'
leer numero 
Mientras  numero  <> trunc(numero) o (numero<1 o numero >499)
	Escribir 'Ingrese un numero entero entre 1 y 499(incluyendo 1 y 499)'
	leer numero
FinMientras
centena<-trunc(numero/100) mod 10
decena <-trunc(numero/10) mod 10
unidad <- trunc(numero/1) mod 10
segun centena Hacer
1:
Escribir 'C' Sin Saltar
2:
Escribir 'CC' Sin Saltar
3:
Escribir 'CCC' Sin Saltar
4:
Escribir 'CD' Sin Saltar
FinSegun
Segun decena hacer

1:
Escribir 'X' Sin Saltar
2:
Escribir 'XX' Sin Saltar
3:
Escribir 'XXX' Sin Saltar
4:
Escribir 'XL' Sin Saltar
5:
Escribir 'L' Sin Saltar
6:
Escribir 'LX' Sin Saltar
7:
Escribir 'LXX' Sin Saltar
9:
Escribir 'XC' Sin Saltar
FinSegun
Segun unidad hacer
1:
Escribir 'I' Sin Saltar
2:
Escribir 'II' Sin Saltar
3:
Escribir 'III' Sin Saltar
4:
Escribir 'IV' Sin Saltar
5:
Escribir 'V' Sin Saltar
6:
Escribir 'VI' Sin Saltar
7:
Escribir 'VII' Sin Saltar
9:
Escribir 'IX' Sin Saltar
FinSegun
FinAlgoritmo
\end{verbatim}
\newpage
\subsection{Ejercicio 2}
\begin{verbatim}
	Algoritmo Ej1_1_2
	Escribir 'Ingrese una serie'
	Leer Serie
	Escribir 'Ingrese los capítulos que posee'
	Leer Capitulos
	Escribir 'Ingrese la duración de cada capítulo en minutos'
	Leer Duración
	dias<-trunc(Capitulos*Duración/(60*24))
	horas<-trunc((Capitulos*Duración/(60*24)-dias)*24)
	minutos<-trunc((((Capitulos*Duración/(60*24)-dias)*24)-horas)*60)
	Escribir 'Te demorarías ',dias,' dias con ',horas,' horas y ',minutos
	,' minutos en ver ',Serie
	FinAlgoritmo
	
\end{verbatim}
\newpage
\section{Respuestas Guía 2}
\subsection{Ejercicio 1}

\begin{verbatim}
	  GNU nano 4.8                                           ej1.sh                                                      #!/bin/bash
	case $1 in
	'Mercurio')
	echo 'El diametro ecuatorial de Mercurio es de 4.878 km'
	'Venus'|'venus'|'VENUS')
	echo 'El diametro ecuatorial de Venus es de 12.100 km'
	;;
	'Tierra'|'tierra'|'TIERRA')
	echo 'El diametro ecuatorial de la Tierra es de 12.756 km'
	;;
	'Marte'|'marte'|'MARTE')
	echo 'El diametro ecuatorial de Marte es de 6.787 km'
	;;
	'Jupiter'|'jupiter'|'JUPITER')
	echo 'El diametro ecuatorial de Jupiter es de 142.984 km'
	;;
	'Saturno'|'saturno'|'SATURNO')
	echo 'el diametro ecuatorial de Saturno es de 120.536 km'
	;;
	'Urano'|'URANO'|'urano')
	echo 'el diametro ecuatorial de Urano es de 51.108 km'
	;;
	'Neptuno'|'neptuno'|'NEPTUNO')
	echo 'el diametro ecuatorial de Neptuno es de 49.538 km'
	;;
	esac
	
\end{verbatim}
\newpage
\subsection{Ejercicio 2}
\begin{verbatim}
	#!/bin/bash
	echo 'Ingrese el número de dureza de la roca'
	read numero
	if [[ $numero -le 2 ]]
	then
	echo 'Roca muy blanda'
	elif [[ $numero -le 3 ]]
	then
	echo 'Roca blanda'
	elif [[ $numero -le 5 ]]
	then
	echo 'Roca Medio blanda'
	elif [[ $numero -le 6 ]]
	then
	echo 'Roca Media dura'
	elif [[ $numero -le 7 ]]
	then
	echo 'Roca Dura'
	elif [[ $numero -le 10 ]]
	then
	echo 'Roca muy Dura'
	else
	echo 'No ha ingresado un número válido'
	fi
\end{verbatim}
\newpage
\section{Respuestas Guía 3}
\subsection{Ejercicio 1}
Se utiliza grep, y -n para indicar las lineas.\\
\begin{verbatim}
grep -n Chile volcano_db.csv > volcanochile.csv
\end{verbatim}
\subsection{Ejercicio 2}
Se utiliza sort, -t"," para indicar el separador, y -k3 para indicar el orden alfabético y columna
\begin{verbatim}
	sort -t"," -k3 volcano_db.csv
\end{verbatim}
\newpage
\section{Respuestas Guía 4}
\subsection{Ejercicio 1}
\lstinputlisting{./Matlab/guia4ej1.m}
\newpage
\subsection{Ejercicio 2}
\lstinputlisting{./Matlab/guia4ej2.m}\newpage
\section{Respuestas Guía 5}
\subsection{Ejercicio 1}
\lstinputlisting{./Matlab/guia5ej1.m}\newpage
\subsection{Ejercicio 2}
\lstinputlisting{./Matlab/guia5ej2.m}\newpage
\section{Respuestas Guía 6}
\subsection{Ejercicio 1}
\lstinputlisting{./Matlab/fibo.m}\newpage
\subsection{Ejercicio 2}
\lstinputlisting{./Matlab/ordenar.m}\newpage
\section{Respuestas Guía 7}
\subsection{Ejercicio 1}
\lstinputlisting{./Matlab/guia7ej1.m}\newpage
\subsection{Ejercicio 2}
\lstinputlisting{./Matlab/guia7ej2.m}\newpage
\section{Respuestas Guía 8}
\subsection{Ejercicio 1}
\lstinputlisting{./Matlab/guia8ej1.m}\newpage
\subsection{Ejercicio 2}
\lstinputlisting{./Matlab/guia8ej2.m}\newpage
\section{Respuestas Guía 9}
\subsection{Ejercicio 1}
\lstinputlisting{./Matlab/guia9ej1.m}\newpage
\subsection{Ejercicio 2}
\lstinputlisting{./Matlab/guia9ej2.m}\newpage
\section{Respuestas Guía 10}
\subsection{Ejercicio 1}
\lstinputlisting{./Matlab/guia10ej1.m}\newpage
\subsection{Ejercicio 2}
\lstinputlisting{./Matlab/guia10ej2.m}\newpage